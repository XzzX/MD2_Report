\chapter{Phasen�bergang}

Nun soll untersucht werden, ob es f�r ein Harte-Kugel-Fluid einen Phasen�bergang gibt. Dazu werden die Partikel in der Startkonfiguration in einem zweidimensionalen dreieckigen Gitter angeordnet. Dies ist die dichteste Anordnung. Als physikalische Dichte wird die Teilchendichte verwendet. Diese ist das Inverse der Fl�che der Elementarzelle und ist gegeben durch:
\begin{equation}
\rho = \frac{2}{\sqrt{3}\Lambda^2}
\end{equation}
wobei $\Lambda$ den Gitterabstand bezeichnet. Da alle Teilchen den gleichen Radius $R=1$ besitzen, ist die maximale Dichte $\rho\approx 0.289$. Diese wird f�r $\Lambda = 2$ erreicht.